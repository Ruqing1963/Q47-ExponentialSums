\documentclass[11pt, a4paper]{article}
\usepackage[utf8]{inputenc}
\usepackage[T1]{fontenc}
\usepackage{amsmath, amssymb, amsthm}
\usepackage{geometry}
\usepackage{cite}
\usepackage{graphicx}
\usepackage{booktabs}
\usepackage{url}
\usepackage[hidelinks]{hyperref}
\usepackage{float}

% Page layout
\geometry{left=3cm, right=3cm, top=3cm, bottom=3cm}

% Theorem environments
\theoremstyle{plain}
\newtheorem{theorem}{Theorem}[section]
\newtheorem{conjecture}[theorem]{Conjecture}
\newtheorem{observation}[theorem]{Observation}

\theoremstyle{definition}
\newtheorem{definition}[theorem]{Definition}
\newtheorem{remark}[theorem]{Remark}

% Title
\title{\textbf{Exponential Sums of the Titan Polynomial:\\[4pt]
Numerical Evidence for Non-Generic Monodromy}}

\author{\textbf{Ruqing Chen}\\
GUT Geoservice Inc., Montreal, Quebec\\
\texttt{ruqing@hotmail.com}}

\date{February 2026}

\begin{document}

\maketitle

\begin{abstract}
We present a numerical study of the exponential sums
$S_p = \sum_{n=0}^{p-1} e^{2\pi i Q(n)/p}$ associated with
the Titan polynomial $Q(n)=n^{47}-(n-1)^{47}$, a degree-46
cyclotomic norm form satisfying the palindromic symmetry
$Q(n) = Q(1-n)$.
The Weil bound gives $|S_p| \le 45\sqrt{p}$, and the sum
decomposes into $45$ Frobenius eigenvalues.
We restrict attention to the $111$ primes $p \equiv 1\pmod{47}$
up to~$50{,}000$, for which the polynomial retains its full root
structure modulo~$p$.
The observed mean normalized magnitude is
$\overline{|S_p/\sqrt{p}|}\approx 1.50$ and the maximum
is~$\approx 8.65$ (at $p=283$).
We compare these statistics with three reference models:
a Gaussian random walk of $45$ unit vectors
(expected mean~$\approx 5.97$),
the compact symplectic group $USp(44)$
(expected mean~$\approx 3.74$),
and the Sato--Tate distribution of $SU(2)$
(applicable to elliptic curves, included as a baseline).
The observed mean lies significantly below all higher-dimensional
predictions, suggesting that the geometric monodromy group is a
proper subgroup of $USp(44)$, constrained by the cyclotomic
origin of~$Q(n)$.
\end{abstract}

\medskip
\noindent\textbf{MSC 2020:} 11L07, 11T23, 11R18

\noindent\textbf{Keywords:} exponential sums, Frobenius eigenvalues,
monodromy group, Katz--Sarnak philosophy, cyclotomic norm form,
Sato--Tate distribution, Weil bound

\section{Introduction}

For a polynomial $f(x)$ of degree $d$ over $\mathbb{F}_p$, the additive exponential sum
\[
S_p(f) = \sum_{n=0}^{p-1} \exp\!\left(\frac{2\pi i f(n)}{p}\right)
\]
encodes deep arithmetic information. By the Weil--Deligne theorem \cite{Deligne1974}, if $\gcd(d, p) = 1$, then
\[
|S_p(f)| \le (d-1)\sqrt{p}.
\]
The sum decomposes as $S_p = -1 + \sum_{j=1}^{d-1} \alpha_j(p)$, where the Frobenius eigenvalues satisfy $|\alpha_j(p)| = \sqrt{p}$.

The Katz--Sarnak philosophy \cite{KatzSarnak} predicts that as $p \to \infty$, the conjugacy class of normalized Frobenius $(\alpha_1/\sqrt{p}, \ldots, \alpha_{d-1}/\sqrt{p})$ becomes equidistributed in a compact group $G$ (the \emph{geometric monodromy group}) according to Haar measure. The distribution of $S_p/\sqrt{p}$ is then determined by the trace distribution of~$G$.

For the Titan polynomial $Q(n) = n^{47} - (n-1)^{47}$ of degree $d = 46$, we have $d-1 = 45$ Frobenius eigenvalues.
A key structural feature is the \emph{palindromic symmetry}
$Q(n) = Q(1-n)$, which implies that the exponential sum satisfies
$S_p = \overline{S_p}$ when $-1$ is a $47$-th power modulo~$p$.
More precisely, by a theorem of Katz~\cite{KatzSarnak}, the
geometric monodromy group of the exponential sum family of a
polynomial~$f$ of degree~$d$ is generically $Sp(d-2)$ when~$f$
has odd degree, or lies in $O(d-1)$ when~$f$ has even degree and
certain additional symmetries hold.
Since $\deg(Q) = 46$ is even but $Q$ arises from the
$47$th~cyclotomic polynomial $\Phi_{47}$ (of odd prime order),
the natural functional equation symmetry is symplectic:
the monodromy group is expected to lie in $USp(44)$
rather than $O(45)$ or $SU(45)$.
This paper presents numerical evidence characterizing the distribution of $S_p/\sqrt{p}$ and its implications for the monodromy group.

\section{Setup and Effective Primes}

\subsection{The Exponential Sum}
We study the normalized variable
\[
x_p = \frac{S_p}{\sqrt{p}}, \quad \text{where} \quad
S_p = \sum_{n=0}^{p-1} \exp\!\left(\frac{2\pi i\, Q(n)}{p}\right).
\]
The Weil bound gives $|x_p| \le 45$.

\subsection{Restriction to $p \equiv 1 \pmod{47}$}

For primes $p \not\equiv 1 \pmod{47}$ (with $p \ne 47$), the map $n \mapsto n^{47}$ is a bijection on $\mathbb{F}_p$ (since $\gcd(47, p-1) = 1$). While the sum $S_p$ is \emph{not} trivial in this case, the bijection creates additional symmetries in the Frobenius eigenvalue structure that complicate the equidistribution analysis.

For $p \equiv 1 \pmod{47}$, the 47th-power map has a non-trivial kernel of order~47, and $Q(n)$ has $\omega_Q(p) = 46$ roots modulo~$p$ \cite{ChenTitan}. All 45 Frobenius eigenvalues are ``fully active'' without degeneracy imposed by the power map.

We therefore restrict to
\[
\mathcal{P}_{\mathrm{eff}} = \{ p \le 50{,}000 : p \equiv 1 \pmod{47} \},
\]
which contains $N = 111$ primes.

\section{Experimental Results}

\subsection{Statistical Summary}

Table~\ref{tab:stats} reports the observed statistics of $|x_p| = |S_p|/\sqrt{p}$ and compares them with three theoretical reference distributions.

\begin{table}[h]
\centering
\caption{Observed statistics of $|S_p|/\sqrt{p}$ compared with
theoretical models for $45$ Frobenius eigenvalues.}
\label{tab:stats}
\begin{tabular}{lcccc}
\toprule
Metric & \textbf{Observed} & Gaussian RW & $USp(44)$ & $SU(2)$\textsuperscript{$\dagger$} \\
\midrule
Mean $\mu$ & $\mathbf{1.50}$ & $\approx 5.97$ & $\approx 3.74$ & $0.85$ \\
Max & $\mathbf{8.65}$ & (unbounded) & 45.0 & 2.0 \\
Weil bound & 45.0 & $\infty$ & 45.0 & 2.0 \\
\bottomrule
\end{tabular}

\smallskip
{\footnotesize $\dagger$ $SU(2)$ applies to elliptic curves
(rank~1, degree~3); included as a pedagogical baseline only.}
\end{table}

Here:
\begin{itemize}
\item \textbf{Gaussian Random Walk}: The modulus of a sum of $45$
independent random unit vectors in~$\mathbb{C}$.
Each vector has expected modulus $\sqrt{\pi/2}\approx 0.89$;
the total sum has expected modulus
$\sqrt{\pi/2}\times\sqrt{45}\approx 5.97$.
This models maximal randomness with no correlations among the
Frobenius eigenvalues.
\item $\boldsymbol{USp(44)}$: The expected trace distribution if
the monodromy group were the full compact symplectic group of
rank~22.
By the Central Limit Theorem for high-rank groups,
$E[|\mathrm{Tr}(U)|] \approx \sqrt{44/\pi} \approx 3.74$.
The palindromic symmetry $Q(n)=Q(1-n)$ provides the structural
reason to expect symplectic rather than unitary monodromy.
\item $\boldsymbol{SU(2)}$: The Sato--Tate distribution
\cite{Taylor2008} with
$E[|a_p|/\sqrt{p}] = 8/(3\pi) \approx 0.85$ and strict bound
$|a_p|/\sqrt{p} \le 2$.
This applies to elliptic curves (degree~3, single Frobenius
eigenvalue pair) and is included only for reference.
\end{itemize}

\begin{observation}[Sub-Generic Mean]
The observed mean $\mu \approx 1.50$ lies well below the $USp(44)$ prediction of $\approx 3.74$. This suggests that the geometric monodromy group of $Q(n)$ is a \emph{proper subgroup} of $USp(44)$, whose trace distribution is more concentrated near the origin.
\end{observation}

\begin{observation}[The $p = 283$ Outlier]
The maximum $|S_p|/\sqrt{p} \approx 8.65$ occurs at $p = 283$,
the smallest prime in $\mathcal{P}_{\mathrm{eff}}$.
At this prime, $Q(n)$ takes only 19 distinct values modulo~283
(out of 283 possible), with the value~0 occurring 46 times.
This extreme concentration of values produces strong constructive
interference.
Excluding $p = 283$, the maximum drops to $\approx 5.97$
(at $p = 6581$) and the mean decreases to $\approx 1.44$.
\end{observation}

\subsection{Distribution of $|S_p|/\sqrt{p}$}

Figure~\ref{fig:dist} shows the four-panel distribution of $S_p/\sqrt{p}$ in the complex plane.

\begin{figure}[H]
\centering
\includegraphics[width=0.95\textwidth]{expsum_figure.pdf}
\caption{Distribution of normalized exponential sums
$S_p/\sqrt{p}$ for the $111$ primes $p \equiv 1\pmod{47}$,
$p \le 50{,}000$.
\textbf{(a)}~Histogram of $\mathrm{Re}(S_p)/\sqrt{p}$ with
standard Gaussian overlay (red dashed).
\textbf{(b)}~Histogram of $\mathrm{Im}(S_p)/\sqrt{p}$ with
Gaussian overlay.
\textbf{(c)}~Scatter plot in the complex plane, color-coded by
prime size; dashed circles at $|S_p|/\sqrt{p}=2,4,6,8$;
the $p=283$ outlier is labelled.
\textbf{(d)}~Histogram of $|S_p|/\sqrt{p}$ with vertical lines
marking the observed mean ($\approx 1.50$, solid),
$USp(44)$ prediction ($\approx 3.74$, dashed), and Gaussian
random walk prediction ($\approx 5.97$, dot-dash).
The concentration of the distribution well below both
higher-dimensional predictions is clearly visible.
(Note: with $N=111$ data points, the histogram shape should
not be over-interpreted; the vertical line positions---i.e.\
the mean---are the robust quantities.)}
\label{fig:dist}
\end{figure}

\section{Discussion}

\subsection{Monodromy Group Constraints}

For a ``generic'' polynomial of degree 46, Katz's
theorem~\cite{KatzSarnak} predicts that the geometric monodromy
group is either $Sp(44)$ or $SU(45)$, depending on the
functional equation symmetry.
The palindromic property $Q(n)=Q(1-n)$ selects the symplectic
case, so $USp(44)$ is the natural generic baseline for~$Q$.

The observed mean of~$\approx 1.50$ is significantly below both
the $USp(44)$ prediction ($\approx 3.74$) and the Gaussian random
walk prediction ($\approx 5.97$ for $45$ independent eigenvalues).
This indicates a degree of cancellation among the Frobenius
eigenvalues far exceeding what either model produces.

A natural explanation is that the associated algebraic variety
undergoes a \emph{Jacobian decomposition}: because
$Q(n) = \Phi_{47}(n/(n-1))$ is built from the $47$th cyclotomic
polynomial, the Jacobian of the associated curve may split into
lower-dimensional abelian varieties with complex multiplication
(CM) by~$\mathbb{Q}(\zeta_{47})$.
If the Jacobian decomposes into~$k$ independent sub-varieties,
the exponential sum becomes a superposition of~$k$ lower-rank
traces, each with smaller variance, reducing the overall
mean magnitude.

\begin{conjecture}[Jacobian Splitting]
The geometric monodromy group of the exponential sum family
$\{S_p(Q)\}$ is a proper subgroup of $USp(44)$, arising from
a nontrivial decomposition of the associated Jacobian variety
into CM abelian sub-varieties indexed by the characters
of~$(\mathbb{Z}/47\mathbb{Z})^\times$.
\end{conjecture}

\subsection{Relation to the Shielding Property}

The shielding property ($\omega_Q(p) = 0$ for $p < 283$)
established in~\cite{ChenTitan} is a statement about the
\emph{zeros} of $Q(n) \pmod{p}$.
The exponential sum $S_p$ encodes all values of
$Q(n) \pmod{p}$, not just the zeros.
Nevertheless, the two phenomena share a common origin: the
factorization $Q(n) = \Phi_{47}(n/(n-1))$ (in projective terms)
ties the arithmetic of $Q$ to the 47th cyclotomic field, imposing
constraints on both the root structure and the Frobenius
eigenvalues.

\section{Conclusion}

Numerical computation of exponential sums for the Titan polynomial
reveals a distribution that is neither Sato--Tate ($SU(2)$) nor
generic symplectic ($USp(44)$).
The observed mean $\overline{|S_p|/\sqrt{p}} \approx 1.50$
lies well below the $USp(44)$ prediction ($\approx 3.74$) and
the uncorrelated random walk prediction ($\approx 5.97$),
suggesting a monodromy group that is a proper, arithmetically
constrained subgroup of the expected symplectic group.

We note that the sample size ($N = 111$ primes) limits the
precision of distributional shape analysis; the mean and
maximum statistics are more robust.
Extending the computation to larger prime ranges, and determining
the precise monodromy group from the cyclotomic structure
of~$Q(n)$, are natural directions for future work.

\medskip
\noindent\textbf{Data availability.}
The \LaTeX{} source, computed data, and SageMath/Python scripts
are available at:
\begin{center}
\url{https://github.com/Ruqing1963/Q47-ExponentialSums}
\end{center}

\begin{thebibliography}{9}

\bibitem{KatzSarnak}
N.\,M. Katz and P. Sarnak,
\textit{Random Matrices, Frobenius Eigenvalues, and Monodromy},
AMS Colloquium Publications, Vol.~45, 1999.

\bibitem{Deligne1974}
P. Deligne,
\textit{La conjecture de Weil.~I},
Publ.\ Math.\ IH\'ES \textbf{43} (1974), 273--307.

\bibitem{Taylor2008}
R. Taylor,
\textit{Automorphy for some $\ell$-adic lifts of automorphic mod~$\ell$ Galois representations.~II},
Publ.\ Math.\ IH\'ES \textbf{108} (2008), 183--239.

\bibitem{ChenTitan}
R.~Chen,
\textit{Prime Values of a Cyclotomic Norm Polynomial and a
Conjectural Bounded Gap Phenomenon},
Preprint (2026),
\url{https://zenodo.org/records/18521551}.

\end{thebibliography}

\end{document}
